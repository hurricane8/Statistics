\documentclass[14pt]{extarticle}

\usepackage[T2A]{fontenc}
\usepackage[utf8]{inputenc}
\usepackage[russian]{babel}
\usepackage{graphicx}
\usepackage{caption}
\DeclareCaptionLabelSeparator{dot}{. }
\captionsetup{justification=centering,labelsep=dot}
\usepackage{amsmath}
\usepackage{amssymb}
\input glyphtounicode.tex
\input glyphtounicode-cmr.tex
\pdfgentounicode=1
\usepackage{bm}

\begin{document}

ФИО: Молодцов Глеб Львович

\vspace{10pt}

Номер задачи: 66(б)

\vspace{10pt}

Решение:

\vspace{10pt}
Запишем матрицу штрафов:
$$
\text{C}=\left(\begin{array}{ll}
0 & c_{12}  \\
c_{12}  & 0
\end{array}\right), \quad \mathbb{P}\left[H_1\right]=\mathbb{P}\left[H_2\right]=\frac{1}{2}
$$
Запишем риски:
$$
R_1(\delta)=0+c_{12}  \cdot \mathbb{P}_1\left[H_2\right]=c_{12} \cdot \alpha, \quad R_2(\delta)=c_{12}  \cdot \beta
$$

Запишем Байесовский критерий для двух гипотез:
$$
l(x)=\frac{f_2(x)}{f_1(x)} \geqslant \frac{c_{21}-c_{11}}{c_{12}-c_{22}} \cdot \frac{q_1}{q_2}=1
$$
Тогда получим Байесовское решающее правило:
$$
\delta(x)=\left\{\begin{array}{l}
1, l(x) \geqslant 1 \\
0, l(x)<1
\end{array}\right.
$$
Пусть решение принимается только на основании измерения второй компоненты. В таком случае СВ будет иметь одно из двух следующих нормальных распределений (одномерных):
$$
H_1^{\prime \prime}: N(0,1) \quad H_2^{\prime \prime}: N(0,1)
$$

Как мы видим, вторые компоненты распределены одинаково.
При этом $l(x) = 1 \Rightarrow$ будет выбираться вторая гипотеза. \\
Найдем ошибки первого и второго рода: 
$\alpha=1, \beta=0$.\\
Минимальный байесовский риск:
$$
r=\frac{\alpha+\beta}{2}\cdot c_{12} =\frac{1}{2} \cdot c_{12} 
$$

Ответ: $r(\delta)= \frac{1}{2} \cdot c_{12} $
\end{document}