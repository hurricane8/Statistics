\documentclass[14pt]{extarticle}

\usepackage[T2A]{fontenc}
\usepackage[utf8]{inputenc}
\usepackage[russian]{babel}
\usepackage{graphicx}
\usepackage{caption}
\DeclareCaptionLabelSeparator{dot}{. }
\captionsetup{justification=centering,labelsep=dot}
\usepackage{amsmath}
\usepackage{amssymb}
\input glyphtounicode.tex
\input glyphtounicode-cmr.tex
\pdfgentounicode=1
\usepackage{bm}

\begin{document}

ФИО: Молодцов Глеб Львович

\vspace{10pt}

Номер задачи: 19

\vspace{10pt}

Решение:

\vspace{10pt}

Допустим, длина стержня обозначается как $l$ и измеренная длина распределена нормально $\mathcal{N}(l, kl)$. Запишем функцию правдоподобия:\\ 
$ L(x, l) = \left(\frac{1}{\sqrt{2\pi kl}}\right)^n\cdot \exp \left(-\frac{1}{2kl}\sum\limits_{i=1}^n (xi - l)^2\right) $

Задача нахождения максимума функции $L$ сводится к задаче нахождения максимума функции $\ln L$. 
\begin{gather*}
    \ln L(x, l) = -\frac{n}{2}\cdot\ln(2\pi kl)-\frac{1}{2kl}\sum\limits_{i=1}^n (x_i - l)^2
    \\
    \frac{\partial \ln L}{\partial l} = 0
    \\
    -\frac{n}{2}\frac{2\pi k}{2\pi kl} - (-1)\frac{1}{2kl^2}\sum\limits_{i=1}^n (x_i - l)^2 - \frac{1}{2kl}(-2)\sum\limits_{i=1}^n (x_i - l) = 0
    \\
    -\frac{n}{2l}+\frac{1}{2kl^2}\sum\limits_{i=1}^n (x_i - l)^2 + \frac{1}{kl}\sum\limits_{i=1}^n (x_i - l) = 0
    \\
    -nkl + \sum\limits_{i=1}^n (x_i - l)^2 + 2l\sum\limits_{i=1}^n (x_i - l) = 0
    \\
    -nkl + \sum\limits_{i=1}^n (x_i^2 - 2x_i l - l^2) + \sum\limits_{i=1}^n (2lx_i - 2l^2) = 0
    \\
    -nkl + \sum\limits_{i=1}^n (x_i^2 - l^2) = 0 
\end{gather*}
Итак,
    $$nl^2 + nkl - \sum\limits_{i=1}^n x_i^2 = 0 $$
    $$l = \frac{-k \pm \sqrt{k^2 + 4\overline{x^2}}}{2}$$
Исследуем на глобальный максимум корень с плюсом.

\begin{gather}
    \frac{\partial^2 \ln L}{\partial^2 l} = \frac{\partial}{\partial l}\left(\frac{-nkl - nl^2 + \sum\limits_{i=1}^n x_i^2}{2kl^2}\right) = 
    \\
    = \frac{-nk - 2nl}{2kl^2} + \frac{-nkl - 2 \sum\limits_{i=1}^n x_i^2 - nl^2}{2kl^3} =
    \\
    = \frac{1}{2kl^3}\left(-nkl - 2nl^2 + 2nkl - 2\sum\limits_{i=1}^n x_i^2 + 2nl^2\right) =\\
    =\frac{1}{2kl^3}\left(nkl - 2\sum\limits_{i=1}^n x_i^2\right)
\end{gather}

Условие на нахождение максимума:
$\frac{\partial^2 \ln L}{\partial^2 l} < 0 \Leftrightarrow \frac{kl}{2} < \frac{1}{n}\sum\limits_{i=1}^n xi^2 = \overline{x^2}.$

Таким образом, при  $kl < 2\overline{x^2} \implies$ 

\vspace{10pt}

Ответ: 
Оценка максимального правдоподобия (ОМП) $l = \frac{-k + \sqrt{k^2 + 4\overline{x^2}}}{2}.$

\end{document}