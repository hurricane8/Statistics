\documentclass[14pt]{extarticle}

\usepackage[T2A]{fontenc}
\usepackage[utf8]{inputenc}
\usepackage[russian]{babel}
\usepackage{graphicx}
\usepackage{caption}
\DeclareCaptionLabelSeparator{dot}{. }
\captionsetup{justification=centering,labelsep=dot}
\usepackage{amsmath}
\usepackage{amssymb}
\input glyphtounicode.tex
\input glyphtounicode-cmr.tex
\pdfgentounicode=1
\usepackage{bm}

\begin{document}

ФИО: Молодцов Глеб Львович

\vspace{10pt}

Номер задачи: 9

\vspace{10pt}

Решение:

\vspace{10pt}

Необходимые условия для применения критерия согласия $\chi ^2 $ Пирсона выполнены. Действительно, $n = 800 \geq 50, \nu = 10 \geq 5$.\\
Для проверки пройдёмся по шагам алгоритма:\\
\begin{itemize}
    \item $\nu_j = \sum\limits_{i=1}^{n} \mathbb{I}\{x_i = j\} $ даны по условию
    \item $T\chi^2 = \sum\limits_{j=1}^N\frac{(\nu_j - np_j^0)^2}{np_j^0} = \frac{36 + 144 + 9 + 1 + 0 + 49 + 9 + 25 + 16 + 121}{80} = \frac{410}{80} \approx 5.125$
    \item $t_\alpha = (1-\alpha) $-квантиль распределения $\chi^2(9)$.\\ \item $t_{0.05} = 16.919  > 5.125 \Rightarrow  $ не отклоняем гипотезу.
\end{itemize}
Ответ: не отклоняем гипотезу

\end{document}