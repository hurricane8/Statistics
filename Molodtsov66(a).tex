\documentclass[14pt]{extarticle}

\usepackage[T2A]{fontenc}
\usepackage[utf8]{inputenc}
\usepackage[russian]{babel}
\usepackage{graphicx}
\usepackage{caption}
\DeclareCaptionLabelSeparator{dot}{. }
\captionsetup{justification=centering,labelsep=dot}
\usepackage{amsmath}
\usepackage{amssymb}
\input glyphtounicode.tex
\input glyphtounicode-cmr.tex
\pdfgentounicode=1
\usepackage{bm}

\begin{document}

ФИО: Молодцов Глеб Львович

\vspace{10pt}

Номер задачи: 66(а)

\vspace{10pt}

Решение:

\vspace{10pt}

Запишем матрицу штрафов:
$$
\text{C}=\left(\begin{array}{ll}
0 & c_{12}  \\
c_{12}  & 0
\end{array}\right), \quad \mathbb{P}\left[H_1\right]=\mathbb{P}\left[H_2\right]=\frac{1}{2}
$$
Запишем риски:
$$
R_1(\delta)=0+c_{12}  \cdot \mathbb{P}_1\left[H_2\right]=c_{12}  \cdot \alpha, \quad R_2(\delta)=c_{12}  \cdot \beta
$$

Запишем Байесовский критерий для двух гипотез:
$$
l(x)=\frac{f_2(x)}{f_1(x)} \geqslant \frac{c_{21}-c_{11}}{c_{12}-c_{22}} \cdot \frac{q_1}{q_2}=1
$$
Тогда получим Байесовское решающее правило:
$$
\delta(x)=\left\{\begin{array}{l}
1, l(x) \geqslant 1 \\
0, l(x)<1
\end{array}\right.
$$
Пусть решение принимается только на основании измерения первой компоненты. В таком случае СВ будет иметь одно из двух следующих нормальных распределений (одномерных):
$$
\tilde{H_1}: N(1,1) \quad \tilde{H_2}: N(-1,1)
$$

Запишем фукнцию $l(x)$ из байесовского правила:
$$
l(x)=\exp \left(\frac{1}{2} \cdot\left(-(x+1)^2+(x-1)^2\right)\right)=\exp (-2x)
$$

Найдем ошибки первого и второго рода:
\begin{gather*}
\alpha=\mathbb{P}_1[l(x) \geqslant 1]=\mathbb{P}_1[\exp (-2 x) \geqslant 1]=\mathbb{P}_1[x \leqslant 0]=F_{N(1,1)}(0)=0,1587 \\
\beta=\mathbb{P}_2[l(x)<1]=\mathbb{P}_2[x>0]=1-F_{N(-1,1)}(0)=0,1587
\end{gather*}

Минимальный байесовский риск:
$$
r(\delta)=q_1 R_1(\delta)+q_2 R_2(\delta)=\frac{\alpha+\beta}{2}=0,1587 \cdot c_{12} 
$$

\end{document}