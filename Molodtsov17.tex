\documentclass[14pt]{extarticle}

\usepackage[T2A]{fontenc}
\usepackage[utf8]{inputenc}
\usepackage[russian]{babel}
\usepackage{graphicx}
\usepackage{caption}
\DeclareCaptionLabelSeparator{dot}{. }
\captionsetup{justification=centering,labelsep=dot}
\usepackage{amsmath}
\usepackage{amssymb}
\input glyphtounicode.tex
\input glyphtounicode-cmr.tex
\pdfgentounicode=1
\usepackage{bm}

\begin{document}

ФИО: Молодцов Глеб Львович

\vspace{10pt}

Номер задачи: 17

\vspace{10pt}

Решение:

\vspace{10pt}

Будем использовать метод моментов для $X_i$.\\
Запишем выборочные моменты: $\hat{\alpha}_m = \frac{1}{n}\sum\limits_{k=1}^n X_k^m$. \\
Тогда $m = \hat{\alpha}_1 = \frac{1}{n}\sum\limits_{k=1}^n X_k = \frac{1}{n}\sum\limits_{k=1}^n \ln V_k = \ln \overline{V_k}$

\end{document}