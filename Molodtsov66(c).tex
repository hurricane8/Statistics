\documentclass[14pt]{extarticle}

\usepackage[T2A]{fontenc}
\usepackage[utf8]{inputenc}
\usepackage[russian]{babel}
\usepackage{graphicx}
\usepackage{caption}
\DeclareCaptionLabelSeparator{dot}{. }
\captionsetup{justification=centering,labelsep=dot}
\usepackage{amsmath}
\usepackage{amssymb}
\input glyphtounicode.tex
\input glyphtounicode-cmr.tex
\pdfgentounicode=1
\usepackage{bm}

\begin{document}

ФИО: Молодцов Глеб Львович

\vspace{10pt}

Номер задачи: 66(в)

\vspace{10pt}

Решение:

\vspace{10pt}

Запишем матрицу штрафов:
$$
\text{C}=\left(\begin{array}{ll}
0 & c_{12} \\
c_{12}  & 0
\end{array}\right), \quad \mathbb{P}\left[H_1\right]=\mathbb{P}\left[H_2\right]=\frac{1}{2}
$$
Для многомерного распределения нужно найти матрицу, обратную к ковариационной матрице:
$$
R^{-1}=\frac{2}{3}\left[\begin{array}{cc}
2 & -1 \\
-1 & 2
\end{array}\right]
$$
Запишем риски:
$$
R_1(\delta)=0+c_{12}  \cdot \mathbb{P}_1\left[H_2\right]=c_{12} \cdot \alpha, \quad R_2(\delta)=c_{12} \cdot \beta
$$

Запишем Байесовский критерий для двух гипотез:
$$
l(x)=\frac{f_2(x)}{f_1(x)} \geqslant \frac{c_{21}-c_{11}}{c_{12}-c_{22}} \cdot \frac{q_1}{q_2}=1
$$
Тогда получим Байесовское решающее правило:
$$
\delta(x)=\left\{\begin{array}{l}
1, l(x) \geqslant 1 \\
0, l(x)<1
\end{array}\right.
$$
Найдём функцию отношения правдоподобия:
\begin{gather*}
    l(x)=\frac{\exp\left(-\frac{1}{2} \cdot\left(x-m_2\right)^T R^{-1}\left(x-m_2\right)\right)}{\exp\left(-\frac{1}{2} \cdot\left(x-m_1\right)^T R^{-1}\left(x-m_1\right)\right)} =
    \\
    = \exp \left(\frac{1}{3} \cdot\left(2x_1^2-2x_1-x_1x_2-2x_1+2+x_2-x_1x_2+x_2+2x_2^2 - \\ - 2x_1^2-2x_1+x_1x_2-2x_1-2+x_2+x_1x_2+x_2-2x_2^2 \right)\right)
    = \\
    =\exp \left(\frac{1}{3} \cdot\left(-8x_1+4x_2 \right)\right) = \exp \left(\left(-\frac{8}{3}x_1+\frac{4}{3}x_2 \right)\right)
\end{gather*}
Найдем ошибки первого и второго рода:

\begin{gather*}
\alpha=\mathbb{P}_1\left[l\left(x\right) \geqslant 1\right]=\mathbb{P}_1\left[\exp \left(-\frac{8}{3}x_1+\frac{4}{3}x_2 \right) \geqslant 1\right]=\mathbb{P}_1\left[-\frac{8}{3}x_1+\frac{4}{3}x_2 \geqslant 0\right]=\\
=\mathbb{P}_1\left[2x_1 - x_2 \leqslant 0\right] \\
\beta=\mathbb{P}_2[l(x)<1]=\mathbb{P}_2[x_2-2x_1<0]
\end{gather*}
Плотность вероятности нормального распределения может быть записана следующим образом:
$$
f_1(x_1, x_2) = \frac{1}{2\pi\sqrt{0.75}} \exp{\left(-\frac{1}{2}\left(\begin{array}{cc}x_1 - 1 & x_2\\ \end{array}\right)\left(\begin{array}{cc}1 & 0.5\\ 0.5 & 1\\ \end{array}\right)^{-1}\left(\begin{array}{c}x_1 - 1\\ x_2\\ \end{array}\right)\right)}.
$$
$$
f_2(x_1, x_2) = \frac{1}{2\pi\sqrt{0.75}} \exp{\left(-\frac{1}{2}\left(\begin{array}{cc}x_1 + 1 & x_2\\ \end{array}\right)\left(\begin{array}{cc}1 & 0.5\\ 0.5 & 1\\ \end{array}\right)^{-1}\left(\begin{array}{c}x_1 + 1\\ x_2\\ \end{array}\right)\right)}.
$$
Теперь интеграл, который нам нужно вычислить, записывается следующим образом:

$$\int\int_D f(x_1, x_2) d(x_1, x_2),$$

где $D$ - область, где выполняется условие $Z = 2x_1 - x_2 \leq 0$.
\begin{gather*}
\alpha=\int_{-\infty}^{\infty} \int_{2x_1}^{\infty} \frac{1}{2\pi \sqrt{0.75}} \exp\Biggl(-\frac{1}{2} \Bigl( \frac{(x_1 - 1)^2}{0.75} - 0.5(x_1 - 1)(x_2 - 0.5) +\\+ \frac{(x_2 - 0.5)^2}{0.75} \Bigr) \Biggr) dx_2 dx_1= 0.180397 \\
\beta=\int_{-\infty}^{\infty} \int_{-\infty}^{2x_1} \frac{1}{2\pi \sqrt{0.75}} \exp\Biggl(-\frac{1}{2} \Bigl( \frac{(x_1 + 1)^2}{0.75} - 0.5(x_1 + 1)(x_2 - 0.5) +\\+ \frac{(x_2 - 0.5)^2}{0.75} \Bigr) \Biggr) dx_2 dx_1= 0.0744975
\end{gather*}

Минимальный байесовский риск:
$$
r(\delta)= 0.180397 + 0.0744975 \approx 0,1274 \cdot c_{12} 
$$
Ответ: $r(\delta)= 0,1274 \cdot c_{12} $
\end{document}