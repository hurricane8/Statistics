\documentclass[14pt]{extarticle}

\usepackage[T2A]{fontenc}
\usepackage[utf8]{inputenc}
\usepackage[russian]{babel}
\usepackage{graphicx}
\usepackage{caption}
\DeclareCaptionLabelSeparator{dot}{. }
\captionsetup{justification=centering,labelsep=dot}
\usepackage{amsmath}
\usepackage{amssymb}
\input glyphtounicode.tex
\input glyphtounicode-cmr.tex
\pdfgentounicode=1
\usepackage{bm}

\begin{document}

ФИО: Молодцов Глеб Львович

\vspace{10pt}

Номер задачи: 22

\vspace{10pt}

Решение:

\vspace{10pt}

Первая статистика: 

$$
A_1^* = \overline{X} - \frac{1}{2}
$$

Необходимо исследовать несмещенность этой статистики. Математическое ожидание статистики равно:

$$
\mathbb E_a \left[\overline{X} - \frac{1}{2}\right] = \mathbb E_a \left[\overline{X}\right] - \frac{1}{2} = \frac{1}{n}n\left(a + \frac{1}{2}\right) - \frac{1}{2} = a 
$$

Следовательно, статистика является несмещенной. Теперь проверим состоятельность:

$$
 T_n(X) \overset{\mathbb P_a}{\underset{n\rightarrow+\infty}{\longrightarrow}} a \Leftrightarrow \underset{n\rightarrow\infty}{\lim} \mathbb P_a\left[|T_n(X) - a| > \varepsilon\right] = 0
$$

Для оценки слагаемого под пределом пользуемся неравенством Чебышева:

$$
\forall \varepsilon > 0 ~\mathbb P\left[|\xi - \mathbb E\xi| > \varepsilon\right] \leqslant \frac{\mathbb D\xi}{\varepsilon^2}
$$

Для $\mathcal{U}[a, a+1]$, получаем:

$$
\mathbb D_a[A_1^*] = \frac{1}{n^2}\mathbb D_a\left[\sum\limits_{i=1}^n X_i\right] = \frac{1}{n^2}\sum\limits_{i=1}^n \mathbb D_a[X_i] = \frac{n}{n^2}\frac{1}{12} = \frac{1}{12n} < +\infty
$$

$$
\forall \varepsilon > 0 ~ \mathbb P_a[|A_1^* - a| > \varepsilon] \leqslant \frac{\mathbb D[A_1^*]}{\varepsilon^2} = \frac{1}{12\varepsilon^2n}\underset{n\rightarrow\infty}{\longrightarrow}0
$$

Получили сходимость по вероятности к нулю, значит $A_1^*$ состоятельна.

Вторая статистика: 

$$
A_2^* = X_n - \frac{n}{n+1}
$$

Перейдем к проверке несмещенности:

$$
\mathbb E_a[A_2^*] = \mathbb E_a[X_n] - \frac{n}{n+1} = \mathbb E_{Beta(n, 1)}[X_n - a] + a - \frac{n}{n+1} = \frac{n}{n+1} + a - \frac{n}{n+1} = a
$$

Таким образом, статистика также является несмещенной. Проверим на состоятельность:

$$
\mathbb D_a[A_2^*] = \mathbb D_a[X_n - a] = \mathbb D_{Beta(n, 1)}[\xi] = 
\frac{n}{(n+1)^2(n+2)}\underset{n\rightarrow\infty}{\longrightarrow}0
$$

Теперь чтобы выяснить, какая статистика предпочтительнее, сравним их дисперсии. 

$$\frac{1}{12n}~ ? ~ \frac{n}{(n+1)^2(n+2)} $$
$$ n^3 - 8n^2 + 5n + 2 ~ ? ~ 0.$$
$$ (n-1)(n-\frac{7}{2} \pm \frac{\sqrt{57}}{2}) ~ ? ~ 0.$$
$$ (n-1)(n+0.27492)(n-7.2749) ~ ? ~ 0.$$

Получили, что при $1\leqslant n < 8$ лучше использовать первую статистику, иначе вторую.
\end{document}