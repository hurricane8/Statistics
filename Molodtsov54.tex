\documentclass[14pt]{extarticle}

\usepackage[T2A]{fontenc}
\usepackage[utf8]{inputenc}
\usepackage[russian]{babel}
\usepackage{graphicx}
\usepackage{caption}
\DeclareCaptionLabelSeparator{dot}{. }
\captionsetup{justification=centering,labelsep=dot}
\usepackage{amsmath}
\usepackage{amssymb}
\input glyphtounicode.tex
\input glyphtounicode-cmr.tex
\pdfgentounicode=1
\usepackage{bm}

\begin{document}

ФИО: Молодцов Глеб Львович

\vspace{10pt}

Номер задачи: 54

\vspace{10pt}

Решение:

\vspace{10pt}

Сначала запишем функцию правдоподобия для обеих гипотез: \\

$$
L_1(x) = \frac{1}{9}\mathbb{I}(x\in[0,3]^2)$$
\\
$$L_2(x) = 0,25 \cdot \exp(-\frac{1}{2}(x_1 + x_2))\cdot \mathbb I\{x_1\geqslant 0\}\mathbb I\{x_2\geqslant 0\}
$$ \\

Теперь составим функцию отношения правдоподобия: \\
$$
l(x) = \frac{0,25\exp(-\frac{1}{2}(x_1 + x_2))\mathbb \mathbb I\{x_1\geqslant 0\}\mathbb I\{x_2\geqslant 0\}}{\frac{1}{9}\mathbb{I}(x\in[0,3]^2)}
$$
Рассмотрим несколько случаев:

\begin{enumerate}
\item $x\in [0,3]^2$. Тогда $l(x) = \frac{9}{4}\exp(-\frac{1}{2}(x_1 + x_2))$.
\item $x_i \in [0,3], x_j\notin[0,3]$. Тогда, если $x_j \geqslant 0$, то $l(x) = +\infty$.
\item $x_i, x_j \notin [0,3]$. Тогда, если $x_i, x_j \geqslant 0$, то $l(x) = +\infty$.
\end{enumerate}

Таким образом,

\begin{equation*}
l(x) = 
\begin{cases}
\frac{9}{4}\exp(-\frac{1}{2}(x_1 + x_2)), x\in[0,3]^2\\
+\infty, x\in [0, +\infty]^2\backslash[0,3]^2
\end{cases}
\end{equation*}

\begin{equation*}
\pi_{c, p}(x) = 
\begin{cases}
1, l(x) > c\\
p, l(x) = c\\
0, l(x) < c
\end{cases}
\end{equation*}

\begin{equation*}
\mathbb P_{H_1}(l(x) > c) + p \cdot \mathbb P_{H_1}(l(x) = c) = \alpha
\end{equation*}
Рассмотрим несколько случаев:
\begin{enumerate}
    \item 

$c<0 \Longrightarrow \mathbb P_{H_1}(l(x) > c) + p \cdot \mathbb P_{H_1}(l(x) = c) \Longrightarrow$ не существует решений уравнения для $\alpha$. 
\item 
 $c>0\Longrightarrow $ 

$\mathbb{P}_{H_{1}}(l(x)>c)= \mathbb{P}_{H_{1}}\left(2.25 \exp \left(-\frac{1}{2}\left(x_{1}+x_{2}\right)\right)>c\right) = \\
= \mathbb{P}_{H_{1}}\left(l(x)=\frac{9}{4} \exp \left(-\frac{1}{2}\left(x_{1}+x_{2}\right)\right)\right)+P_{H_{1}}(l(x)=+\infty)= \\
=\mathbb{P}_{H_{1}}\left(\frac{9}{4} \exp \left(-\frac{1}{2}\left(x_{1}+x_{2}\right)\right)>c\right)
=\mathbb{P}_{H_{1}}\left(-\left(x_{1}+x_{2}\right)>2 \ln \left(\frac{4}{9} c\right)\right)= \\ =\mathbb{P}_{H_{1}}\left(x_{1}+x_{2}<c_1\right)
$

Рассмотрим площадь под пересечением прямой и квадрата. Найдём какое-нибудь $c_1$: такое, когда данная фигура - треугольник.

$$
\mathbb{P}_{H_{1}}\left(x_{1}+x_{2}<c_1\right)=\frac{1}{2} c_1^{2}
$$

Аналогично:

$$
\mathbb{P}_{H_{1}}(l(x)=c)=\mathbb{P}_{H_{1}}\left(x_{1}+x_{2}=c_1\right)=0
$$

Так как $\mathbb{P}_{H_{1}}\left(x_{1}+x_{2}=c_1\right)=0$, в силу того, что $x_{1}+x_{2}$ распределено непрерывно.

Тогда получим:

$
\frac{1}{2} c_1^{2}=\alpha \Longrightarrow c_1=2 \sqrt{\alpha} \Longrightarrow p$ получилось произвольным $\Longrightarrow p=0$.
\end{enumerate}
В итоге

\begin{equation*}
\pi(x) = 
\begin{cases}
1, x_{1}+x_{2}<2 \sqrt{\alpha}, \text { или } x_{1}>3, \text { или } x_{2}>3\\
0, x_{1}+x_{2} \geq 2 \sqrt{\alpha}
\end{cases}
\end{equation*}

$
\beta(\pi)=1-\mathbb{E}_{2} \pi(X)=1-\mathbb{P}_{2}\left(x_{1}+x_{2}<2 \sqrt{\alpha}\right)-\mathbb{P}_{2}\left(x_{1}>3\right)- \\ -\mathbb{P}_{2}\left(x_{2}>3\right) 
=1-\Gamma(2,2)_{2 \sqrt{\alpha}}-\left(1-1+\exp \left(-\frac{3}{2}\right)\right)-\left(1-1+\exp \left(-\frac{3}{2}\right)\right) \Rightarrow 

\\ 
\Rightarrow 
\beta(\pi) = 0.545
$


\end{document}