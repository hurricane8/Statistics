\documentclass[14pt]{extarticle}

\usepackage[T2A]{fontenc}
\usepackage[utf8]{inputenc}
\usepackage[russian]{babel}
\usepackage{graphicx}
\usepackage{caption}
\DeclareCaptionLabelSeparator{dot}{. }
\captionsetup{justification=centering,labelsep=dot}
\usepackage{amsmath}
\usepackage{amssymb}
\input glyphtounicode.tex
\input glyphtounicode-cmr.tex
\pdfgentounicode=1
\usepackage{bm}

\begin{document}

ФИО: Молодцов Глеб Львович

\vspace{10pt}

Номер задачи: 52

\vspace{10pt}

Решение:

\vspace{10pt}

Запишем матрицу штрафов:
\begin{equation*}
\text{C} = \left(
\begin{array}{cccc}
-1 & 2 \\
2 & -1 
\end{array}
\right)
\end{equation*}
$$
\pi_{c, p}(x)=\left\{\begin{array}{l}
1, l(x)>c \\
p, l(x)=c \\
0, l(x)<c
\end{array}\right.
$$
Выпишем риски:
$$
\begin{aligned}
& R_1(\pi)=-1 \cdot \mathbb{P}_1\left[H_1\right]+2 \cdot \mathbb{P}_1\left[H_2\right]=-1 \cdot (1 - p_{21}) +2 \cdot p_{21} =3 \alpha-1 \\
& R_2(\pi)= 2 \cdot \mathbb{P}_2\left[H_1\right] -1 \cdot \mathbb{P}_2\left[H_2\right]=2 \cdot p_{12} -1 \cdot (1 - p_{12})=3 \beta-1
\end{aligned}
$$
Таким оборазом, $\alpha = \beta$. \\
$$
\left.
  \begin{array}{ccc}
    R_1(\pi) = -1 \cdot (1 - p_{21}) +2 \cdot p_{21} = 3 p_{21} - 1 \\
    R_2(\pi) = p_{12} -1 \cdot (1 - p_{12}) =3 p_{12} - 1 \\
  \end{array}
\right\} \Rightarrow 3 p_{21} - 3 p_{12} = 2
$$

$$
3\left(\mathbb{P}_1\left\{l\left(x\right)>c \right\} + p \mathbb{P}_1\left\{l\left(x\right)=c \right\}\right) + 3\left(\mathbb{P}_2\left\{l\left(x\right)>c \right\} + p \mathbb{P}_2\left\{l\left(x\right)=c \right\}\right) = 2
$$
Функция отношения правдоподобия:
$$
l\left(x\right)=\frac{f_2(x)}{f_1(x)}= \left\{\begin{array}{l}
\frac{4 \cdot \exp \left(-2\left(x_1 + x_2\right)\right)}{\frac{1}{2 \pi} \cdot \exp\left(-\frac{\left(x_1+1\right)^2+\left(x_2+1\right)^2}{2}\right)}, x_1, x_2 \geqslant 0 \\
0, \text { иначе}
\end{array}\right. \Longrightarrow
$$
Получим, что
$$
l\left(x\right)=\frac{f_2(x)}{f_1(x)}= \left\{\begin{array}{l}
8 \pi \cdot \exp\left(\frac{\left(x_1-1\right)^2+\left(x_2-1\right)^2}{2}\right), x_1, x_2 \geqslant 0 \\
0, \text { иначе }
\end{array}\right.
$$
Подставим одно из граничных значений для получения следующего критерия:
$$
\pi=\pi_{c, 1}=\left\{\begin{array}{l}
1, l\left(x_1, x_2\right) \geqslant c \\
0, l\left(x_1, x_2\right)<c
\end{array} \quad \Rightarrow c=\frac{c_{21}-c_{11}}{c_{12}-c_{22}} \cdot \frac{q_1}{q_2}=\frac{2+1}{2+1} \cdot \frac{0,6}{0,4}=\frac{3}{2}\right.
$$
Тогда
$$
\pi=\pi_{c, 1}=\left\{\begin{array}{l}
1, l\left(x_1, x_2\right) \geqslant 1.5 \\
0, l\left(x_1, x_2\right)<1.5
\end{array} \right.
$$

Найдем ошибки первого и второго родов.
\begin{gather*}
\alpha=\mathbb{P}_1\left[l(x) \geqslant \frac{3}{2}\right]= \\
=\mathbb{P}_1\left[x_1 \geqslant 0, x_2 \geqslant 0,\left(x_1-1\right)^2+\left(x_2-1\right)^2 \geqslant 2 \cdot \ln \left(\frac{3}{16 \pi}\right)\right]= \\
=\mathbb{P}_1\left[x_1 \geqslant 0, x_2 \geqslant 0,\left(x_1-1\right)^2+\left(x_2-1\right)^2 \geqslant-5.63\right] =\\
=\mathbb{P}_1\left[x_1 \geqslant 0, x_2 \geqslant 0\right]=\left(1-F_{N(-1,1)}^2(0)\right)^2=(1-0.841)^2=0.025
\\
\\
\beta=\mathbb{P}_2\left[l\left(x_1, x_2\right)<\frac{3}{2}\right]=\\
=\mathbb{P}_2\left[\left(x_1-1\right)^2+\left(x_2-1\right)^2<-5.63\right]+\mathbb{P}_2\left[x_1<0 \text { или } x_2<0\right]=\\
=0 + 0 = 0
\end{gather*}
Ответ:
\vspace{10pt}
$$ \alpha = 0.025, \quad \beta = 0$$

\end{document}