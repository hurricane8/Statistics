\documentclass[14pt]{extarticle}

\usepackage[T2A]{fontenc}
\usepackage[utf8]{inputenc}
\usepackage[russian]{babel}
\usepackage{graphicx}
\usepackage{caption}
\DeclareCaptionLabelSeparator{dot}{. }
\captionsetup{justification=centering,labelsep=dot}
\usepackage{amsmath}
\usepackage{amssymb}
\input glyphtounicode.tex
\input glyphtounicode-cmr.tex
\pdfgentounicode=1
\usepackage{bm}

\begin{document}

ФИО: Молодцов Глеб Львович

\vspace{10pt}

Номер задачи: 72

\vspace{10pt}

Решение:

\vspace{10pt}

Запишем матрицу штрафов:
\begin{equation*}
\text{C} = \left(
\begin{array}{cccc}
0 & 1 \\
1 & 0 
\end{array}
\right)
\end{equation*}
$$
\pi_{c, p}(x)=\left\{\begin{array}{l}
1, l(x)>c \\
p, l(x)=c \\
0, l(x)<c
\end{array}\right.
$$
Выпишем риски:
$$
\begin{aligned}
& R_1(\pi)=0 \cdot \mathbb{P}_1\left[H_1\right]+1 \cdot \mathbb{P}_1\left[H_2\right]= p_{21} = \alpha\\
& R_2(\pi)= 1 \cdot \mathbb{P}_2\left[H_1\right] -0 \cdot \mathbb{P}_2\left[H_2\right]= p_{12} =\beta
\end{aligned}
$$
Таким оборазом, $\alpha = \beta$. \\

Функция отношения правдоподобия:
$$
l\left(x\right)=\frac{f_2(x)}{f_1(x)}= \frac{I\{x \in [1;3]^n\}}{I\{x \in [0;2]^n\}}= \left\{\begin{array}{l}
+ \infty , x \in [1;3]^n \And  \exists i : x_i \in (2;3]^n\\
1 , x \in [1;2]^n\\
0 , x \in [0;2]^n \And  \exists i : x_i \in [0;1)^n\\
\text{не определена , иначе}
\end{array}\right. \Longrightarrow
$$
Подставим $c=1$:
$$
\alpha = \mathbb{P}_1[l(x)>c]+p\cdot\mathbb{P}_1[l(x)=c]=p\cdot \mathbb{P}_1[x \in [1,2]^n] = \frac{p}{2^n}
$$
$$
\beta = \mathbb{P}_2[l(x)<c]+(1-p)\cdot\mathbb{P}_2[l(x)=c]=(1-p)\cdot \mathbb{P}_2[x \in [1,2]^n] = \frac{1-p}{2^n}
$$

$$
\left.
  \begin{array}{ccc}
    R_1(\pi) = \frac{p}{2^n}\\
    R_2(\pi) = \frac{1-p}{2^n} \\
  \end{array}
\right\} \Rightarrow p = 0.5
$$

Получим следующий критерий:
$$
\pi_{c, p}(x)=\left\{\begin{array}{l}
1, l(x)>1 \\
0.5, l(x)=1 \\
0, l(x)<1
\end{array}\right.
$$

\end{document}