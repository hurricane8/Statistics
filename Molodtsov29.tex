\documentclass[14pt]{extarticle}

\usepackage[T2A]{fontenc}
\usepackage[utf8]{inputenc}
\usepackage[russian]{babel}
\usepackage{graphicx}
\usepackage{caption}
\DeclareCaptionLabelSeparator{dot}{. }
\captionsetup{justification=centering,labelsep=dot}
\usepackage{amsmath}
\usepackage{amssymb}
\input glyphtounicode.tex
\input glyphtounicode-cmr.tex
\pdfgentounicode=1
\usepackage{bm}

\begin{document}

ФИО: Молодцов Глеб Львович

\vspace{10pt}

Номер задачи: 29

\vspace{10pt}

Решение:

\vspace{10pt}

Рассмотрим вероятность $L(x, \theta) = \frac{\mathbb{I}\{x \in [\theta, 2\theta]^n\}}{\theta^n}$. \\
Таким образом, $\theta \leq X_{(1)} , X_{(n)} \leq 2\theta \Rightarrow L(x, \theta) = \frac{\mathbb{I}\{X_{(1)} \geq \theta\}\cdot \mathbb{I}\{X_{(n)} \leq 2\theta\}}{\theta^n}$ \\
Тогда рассмотрим статистику $S(x) = \left(X_{\left(1\right)}\quad X_{\left(n\right)}\right) $ \\
В данном случае статистика является достаточной по критерию факторизации, так как для нее можно ввзять $g(s, \theta) = \frac{\mathbb{I}\{X_{(1)} \geq \theta\}\cdot \mathbb{I}\{X_{(n)} \leq 2\theta\}}{\theta^n}$, её размероность равна 2.
\end{document}