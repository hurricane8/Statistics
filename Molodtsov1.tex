\documentclass[14pt]{extarticle}

\usepackage[T2A]{fontenc}
\usepackage[utf8]{inputenc}
\usepackage[russian]{babel}
\usepackage{graphicx}
\usepackage{caption}
\DeclareCaptionLabelSeparator{dot}{. }
\captionsetup{justification=centering,labelsep=dot}
\usepackage{amsmath}
\usepackage{amssymb}
\input glyphtounicode.tex
\input glyphtounicode-cmr.tex
\pdfgentounicode=1
\usepackage{bm}

\begin{document}

ФИО: Молодцов Глеб Львович

\vspace{10pt}

Номер задачи: 1

\vspace{10pt}

Решение:

\vspace{10pt}
Запишем случайной величины для эмпирической функции распределения:
\begin{equation}
    \notag
    \frac{nF_n(x^*)-nF(x^*)}{\sqrt{nF(x^*) \left( 1-F \left(x^* \right) \right)}} \xrightarrow[n \rightarrow \infty]{d} \xi_{\mathcal{N}(0,1)} \in \mathcal{N}(0,1)
\end{equation}
Сведём вероятность, данную в задаче, к левой части Центральной предельной теоремы:
\begin{multline*}
    \mathbb{P} \left\{ |F_n(x^*)-F(x^*)| \leqslant \frac{t}{\sqrt{n}}\right\} = \\ =
    \mathbb{P}\left\{\frac{n|F_n(x^*)-F(x^*)|}{\sqrt{n F(x^*) \left( 1-F \left(x^* \right) \right)}} \leqslant \frac{t}{\sqrt{F(x^*) \left( 1-F \left(x^* \right) \right)}}\right\} = \\
    = \mathbb{P}\biggl\{ - \frac{t}{\sqrt{F(x^*) \left( 1-F \left(x^* \right) \right)}}
    \leqslant \frac{nF_n(x^*)-nF(x^*)}{\sqrt{n F(x^*) \left( 1-F \left(x^* \right) \right)}} \leqslant \\
    \leqslant \frac{t}{\sqrt{F(x^*) \left( 1-F\left(x^* \right) \right)}} \biggr\} 
\end{multline*}
Вспомним плотность функции распределения случайной величины из нормального распределения $\xi \in \mathcal{N}(0,1)$:
\begin{equation}
    \notag
    f(x) = \frac{1}{\sqrt{2 \pi}} \exp\left(-\frac{x^2}{2} \right) \Rightarrow F_\xi(x) = \int_{-\infty}^{x} f(y) \,dy 
\end{equation} 
Тогда найдем итоговый ответ, взяв определенный интеграл с полученными из ЦПТ верхним и нижним пределами:
\begin{multline*}
    \mathbb{P} \left\{ |F_n(x^*)-F(x^*)| \leqslant \frac{t}{\sqrt{n}}\right\} = \\
    = F_\xi \left(\frac{t}{\sqrt{F(x^*) \left( 1-F \left(x^* \right) \right)}} \right) - F_\xi\left(- \frac{t}{\sqrt{F(x^*) \left( 1-F \left(x^* \right) \right)}} \right) ,\xi \in \mathcal{N}(0,1)
\end{multline*}
\end{document}